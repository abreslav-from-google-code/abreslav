\section{Hamiltonian path and circuit} \label{algo}

Paper \cite{Katsanyi:2003} represents several algorithms for NP-complete problems implemented in \emph{PFM}. Here we examine complexity of two of these algorithms -- Hamiltonian path and circuit -- according to proposed parallel implementations of basic operations.

\subsection{Notation preliminaries}
Here we provide brief overview of notation used in \cite{Katsanyi:2003}. 

There is an alphabet
$$\Sigma_n = \{p_1, p_2, \dots, p_n, a_1, a_2, \dots, a_n\}$$
and a set
$$T_n = \{p_1 a_{i_1} p_2 a_{i_2} \dots p_n a_{i_n} |
1 \leq i_j \leq n (j = 1, 2, \dots, n)\}.$$
The indexed $p$ symbols denote the \emph{positions} within a word and the indexed $a$ symbols denote integer values between $1$ and $n$. Integers are always separated by position markers.

Many algorithms operate on initial multiset 
$$P_n=\{p_1 a_{i_1} p_2 a_{i_2} \dots p_n a_{i_n} \in T_n | 
         1 \leq j, j' \leq n, j \neq j' \Rightarrow i_j \neq i_{j'}\}$$ 
which represents all permutations of $n$ distinct things. This multiset can be generated using basic operations in polynomial time \cite{Amos:1996} but actually since wide range of algorithms uses it, it might be useful to generate many instances of $P_n$ in advance and then use those instances without spending time on it's generation.

\subsection{Hamiltonian path problem}
For a graph $G=(V, E)$, where $V=\{v_1, \dots, v_n\}$, node sequences are represented by strings in $P_n$. The string $p_1 a_{i_1} p_2 a_{i_2} \dots p_n a_{i_n}$ represents the node sequence $v_{i_1}, v_{i_2}, \dots, v_{i_n}$. For example such a node sequence is a Hamiltonian path if and only if it is a path. That is, the consecutive nodes in the sequence must be connected by an edge in $E$.

Thus, the following function from \cite{Amos:1996} finds Hamiltonian path in a graph.

\function{$HamiltonianPath(V, E)$}{
  \STATE $T \gets P_n$
  \STATE $Remove(T, \{a_j p_i a_{j'} | 2 \leq i \leq n, 1 \leq j, j' \leq n \mbox{ and } \{j, j'\} \notin E\})$
  \RETURN{$Select(T)$}
}

Paper \cite{Katsanyi:2003} shows that when non-parallel implementation of $Remove$ is used, this algorithm's time complexity is $O(n^3)$, since this is cardinality of the set of strings by which $T$ is being filtered. If parallel implementation is used, this complexity reduces to $O(log(n^3))=O(log(n))$ -- complexity of $Remove(T, \{s_1, \dots, s_{n^3}\})$.

\subsection{Hamiltonian circuit problem}

Given an $n$-node undirected graph $G = (V, E)$, where $V=\{v_1, \dots, v_n\}$ find a Hamiltonian circuit (a simple circuit that includes all vertices of $G$).

The following function is quoted from \cite{Katsanyi:2003}.

\function{$HamiltonianCircuit(V, E)$}{
	\STATE $T \gets P_n$
	\STATE $Remove(T, \{a_j p_i a_{j'} 
	   | 2 \leq i \leq n, 1 \leq j, j' \leq n 
	   \mbox{ and } \{v_j, v_{j'}\} \notin E\})$ \label{hc_remove}
	\STATE $Copy(T, \{T_1, \dots, T_n\})$\
	\FOREACH{$i = (1, 2, \dots, n)$ \INPARALLEL}
		\STATE $Remove(T_i, \{p_1 a_j
		  |1 \leq j \leq n \mbox{ and } j \neq i\})$ \label{hc_r1}
		\STATE $Remove(T_i, \{p_n a_j
		  |1 \leq j \leq n \mbox{ and } \{v_i, v_j\} \notin E\})$ \label{hc_r2}
	\ENDFOR
	\STATE $Union(\{T_1, \dots, T_n\}, T)$
	\RETURN{$Select(T)$}
}

Correctness of this algorithm is proved in \cite{Katsanyi:2003}. Time complexity in non-parallel model is $O(n^3)$, the longest operation is line \ref{hc_remove} since cardinality of set of subsrtings by which $T$ is being filtered is $O(n^3)$.

Let us analyze time complexity of this algorithm using proposed parallel implementation of basic operations. Then line \ref{hc_remove} takes $O(log(n^3))=O(log(n))$ parallel time. So does the following line. Lines \ref{hc_r1} and \ref{hc_r2} take $O(log(n))$ parallel time each. So the whole {\bf for} does since it is executed in parallel. Final $Union$ takes $O(log(n))$ parallel time and $Select$ is executed in constant time. Thus, time complexity of this algorithm with parallel implementations of basic operations is $O(log(n))$.

\subsection{Non-time aspects of complexity}
Now let us look at other aspects of the algorithm's complexity. Number of DNA strands used by the algorithm is $O(n!)$ in both parallel and non-parallel implementations.

Number of elementary steps performed by this algorithm is still $O(n^3)$.

We also can examine a metric that can be thought about as \emph{number of tubes}. Maybe tubes are not a really critical resource but this metric allows to measure number of laboratory equipment (other than tubes) used to execute the algorithm. 

How many tubes do we need for $HamiltonianCircuit$ procedure? Each of $Copy(T, \{T_1, \dots, T_n\}$, $Union(\{T_1, \dots, T_n, T\}$ and $Intersect(\{T_1, \dots, T_n, T\}$ need $O(n)$ tubes which form a binary tree with $n$ leaves. $Remove(T, \{s_1, \dots, s_n\})$ consists of $Copy$ and $Intersect$ and thus needs $O(n)$ tubes too.

A call to $Remove$ on line \ref{hc_remove} needs $O(n^3)$ tubes, other operations need $O(n)$ tubes each. Thus ``tube-complexity'' of $HamiltonianCircuit$ procedure is $O(n^3)$. 

However we can reduce amount of tubes needed but we must pay for this with execution time. We can restrict $Remove$ implementation to use not more than $M$ tubes. Let us call this new operation $Remove_M(T, S)$. Having only $M$ tubes we can build a binary tree with $\lceil M/2 \rceil$ leaves, so we simply take first $\lceil M/2 \rceil$ items from set $S$ and filter $T$ by them. Then we take next $\lceil M/2 \rceil$ and do the same and so on.

\procedure{$Remove_M(T, \{s_1, \dots, s_n\})$}{
	\STATE $count \gets 0$
	\WHILE{$count < n$}
		\STATE $Remove(T, \{s_{count+1}, \dots, s_{min\{count+M, n\}}\})$
		\STATE $count \gets count + M$
	\ENDWHILE
}

This results in $O\left(\frac{n \cdot log(M)}{M}\right)$ parallel time. So if we restrict the algorithm to use $O(n)$ tubes, we will use $Remove_n$ instead of $Remove$ on line \ref{hc_remove} and this line will take $O( n^2 log(n))$ parallel time, which is still better when non-parallel version while using asymptotically the same number of tubes.