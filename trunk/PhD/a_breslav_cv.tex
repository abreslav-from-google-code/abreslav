\documentclass[a4paper, 12pt]{article}
\usepackage[left=2cm,right=2cm,top=3cm,bottom=3cm]{geometry}
\usepackage{hyperref}
\pagenumbering{none}

\newcommand{\AMSE}{\href{http://amse.ru/about.php}{AMSE}}
\newcommand{\SwiftTeams}{\href{http://swiftteams.com}{\hyphenation{SwiftTeams}{SwiftTeams}}}
\newcommand{\OpenWay}{\href{http://openwaygroup.com}{OpenWay}}
\newcommand{\JetBrains}{\href{http://jetbrains.com}{JetBrains}}
\newcommand{\DoRa}{\href{http://www.archimedes.ee/amk/File/DoRa_PROGRAMM_eng.doc}{DoRa}}
\newcommand{\Archimedes}{\href{http://www.archimedes.ee/index.php?language=2}{Archimedes Foundation}}
\newcommand{\ERDF}{\href{http://ec.europa.eu/regional_policy/funds/feder/index_en.htm}{European Regional Development Funds}}
\newcommand{\UMNIK}{\href{http://www.fasie.ru/fund_programms/umnik/umnik-index.aspx}{UMNIK}}
\newcommand{\FASIE}{\href{http://www.fasie.ru}{FASIE Foundation}}
\newcommand{\DigDes}{\href{http://www.digdes.com/}{Digital Design}}
\newcommand{\Borland}{\href{http://borland.com}{Borland Labs Inc.}}
\newcommand{\STACC}{\href{http://stacc.ee}{STACC}}
\newcommand{\bl}[1]{
\begin{tabular}[t]{lp{.77\textwidth}}
$\circ$& #1 \\
\end{tabular}
}

\begin{document}
\section*{Andrey Breslav}

\subsection*{Contact Information}
\begin{tabular}{rll}
	Address &\hspace{10pt}& Udarnikov pr., 27-2-237, 195279 Saint-Petersburg, Russia\\
	Phone (mobile) && +7 921 928 67 68 or +372 581 12 831\\
	E-Mail && abreslav@gmail.com \\
	Home page && \url{http://abreslav.googlepages.com}\\
\end{tabular}

\subsection*{Education}

\begin{tabular*}{1.0\textwidth}[t]{p{70pt} l p{370pt}}
	\raggedleft{since 2007}&\hspace{10pt}&PhD student at Mathematics Department, Saint-Petersburg State University of Information Technology, Mechanics and Optics (ITMO)\\
	&&Advisor: Prof. Igor Yu. Popov \\
	&&Thesis submission (planned): May 1st, 2010\\
	&&Defence (planned): not later than September 2010\\
	&&Topic: \textit{Automatically Derived Languages and Applications to DSLs with Aspects}\\
	&&Our goal is to provide a formal framework for automatically deriving infrastructural functionality 
	  from specifications of domain-specific languages (DSLs).
	  As an application, we consider derivation of languages of typed macros (templates) and structural patterns
	  which we use to implement parameterized modules and aspect-oriented features in DSLs. \\
	&&\\
	\raggedleft{2009 -- 2010}&& Visiting PhD student at Institute of Computer Science, University of Tartu, Estonia\\
	&&Advisor: Prof. Marlon Dumas \\
	&&\\
	\raggedleft{2007} && Master's Degree with Honors in Applied Mathematics and Informatics, \\
	&&Computer Technology Department, ITMO, GPA 4.90/5.0\\
	&&Thesis: \textit{DSL Development Based on Target Meta-models}\\
	&&The main contribution is a declarative language for describing an abstract syntax tree (AST) of a 
	  DSL as a view of its target meta-model. From these descriptions we generate translators. \\
	&&\\
	\raggedleft{2005} && Bachelor's Degree with Honors in Applied Mathematics and Informatics, \\
	&&Computer Technology Department, ITMO, GPA 4.96/5.0\\
	&&Thesis: \textit{Automated Generation of User Interfaces from Domain Models}\\
\end{tabular*}
\subsection*{Research Interests}
\begin{itemize}
	\setlength{\itemsep}{0pt}
	\item Declarative specifications of Domain-Specific Languages
		\begin{itemize}
			\item Reuse and modularity of grammar-based specifications
			\item Extensibility of textual languages
		\end{itemize}
	\item Formal semantics of transformations between text and models
	\item Type systems for software languages
\end{itemize}



\subsection*{Teaching}
\begin{tabular}[t]{ r l p{395pt} }
	2006 -- 2009 && {\it Lecturer} at ITMO University and \AMSE{}\\%
	Courses: && \bl{Software Design}\\
		&& \bl{Introduction to Programming}\\
		&& \bl{Algorithms and Data Structures}\\
		&& \bl{Java Programming Language (Basic and Advanced)}\\
		&& \bl{Supervision of students' Software Projects}\\
	 Supervision:&& Master thesis \\
	 &&\bl{\textit{Static Thread Types Analysis for Access Management in Multi-Threaded Java Programs}}\\
		&&Bachelor theses\\
		&& \bl{\textit{Implementation of Declarative QVT Transformations using Maude}}\\
		&& \bl{\textit{Extending JVM Instruction Set with Unchecked Method Calls for Speeding Up Execution on Mobile Platforms}}\\
		&& \bl{\textit{Graph-based modelling, Generation and LTL$-$Verification of Data-Oriented User Interfaces}}\\
		&&\\
	summer 2008 && {\it Summer internship supervisor} at software companies \OpenWay{} and \SwiftTeams{} / \JetBrains{} (as a part of \AMSE{} project)\\%
		&&\\
		 2003 -- 2009 && {\it Teacher of Informatics} at High School 239 (Lyceum with emphasis on Physics and Mathematics), Saint-Petersburg, Russia\\%
\end{tabular}

\subsection*{Software development}
\begin{tabular}{ r l p{395pt} }
\multicolumn{3}{l}{\textbf{Experience}}\\
	2010 && \STACC{}, R\verb & D Engineer in ``String-Embedded DSLs'' project. Working on static analyses of Java programs which generate other programs (i.e. SQL queries), applying methods from language and automata theory combined with static program analyses such as abstract interpretation.\\
	2005 -- 2006 && \Borland{}, R\verb & D Engineer (Java developer) in Together for Eclipse project, MDA team: worked on QVT language implementation, Eclipse plug-in development and integration, EMF modeling, using QVT and XSL/OCL in generative programming samples. \\
2004 -- 2005 && Freelance web development (PHP, MySQL).\\
2003 && UI development (Delphi) for a suite for automatically planning routs for air photography.\\
&&\\
\multicolumn{1}{l}{\textbf{Skills}}
&& Java, Scala, C++, Haskell, Pascal/Delphi, PHP, JavaScript, SQL, XML, XSLT/XPath, HTML/CSS, \LaTeX, Linux, Windows
\end{tabular}

\newpage
\subsection*{Grants and Scholarships}
\begin{itemize}
	\item \textit{\DoRa{} 5} scholarship for visiting PhD studies at University of Tartu from \textit{\ERDF{}} through \textit{\Archimedes{}}, 2009
	\item An ``\textit{\UMNIK{}}'' grant for innovative project development from the {\it \FASIE{}}, 2009
		\begin{itemize}
			\item for the project ``Aspect-based DSL development technology'' 
		\end{itemize}
	\item A grant from {\it The Annual Grant Competition for Young Researchers}, Saint-Petersburg, 2008
	\item {Teaching support grants from 
		\begin{itemize}
			\item \DigDes{} (2003--2004)  
			\item \Borland{} (2004--2005) 
		\end{itemize}}
\end{itemize}

\subsection*{Languages}

\begin{itemize}
	\item Russian -- native
	\item English -- fluent
	\item Estonian -- basic
\end{itemize}

\subsection*{Publications}
\begin{itemize}
	\item {\it Grammatical Aspects for Language Descriptions}, submitted to LDTA-2010
	\item {\it Creating Textual Language Dialects Using Aspect-like Techniques}, In Pre-Proceedings of GTTSE-2009, Braga, 2009 (extended abstract)
	\item {\it Techniques for Formal Grammar Reuse and Their Applications for Creating Dialects}, In Proceedings of KMU-2009, Saint-Petersburg, Russia, April, 2009 (In Russian, abstract in English)
	\item {\it Applying MDD and AOP Principles to Grammarware development}, In Proceedings of KMU-2008, Saint-Petersburg, Russia, April, 2008 (In Russian, abstract in English)
	\item with A. Lukianova, M. Korotkov, {\it Building Class Hierarchies from Informal English Descriptions}, presented at JASS-2004, published in Nauchno-Technicheskiy Vestnik SPbSU ITMO, Vol 39., 2007, pp. 294-303 (In Russian, abstract in English)
	\item with A. Efremov, M. Korotkov, {\it Improving Distance Learning of the Humanities Using {\it RemEd} Online system}, In Proceedings of The 6th IST/IMS-2003 Conference, Saint-Petersburg, Russia, 2003 (In Russian, abstract in English).
\end{itemize}
\end{document}
