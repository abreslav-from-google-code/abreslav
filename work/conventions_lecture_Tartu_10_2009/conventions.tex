\documentclass[screen]{beamer}
\usepackage{listings}
\usepackage{color}
\usepackage{bold-extra}
\usepackage{wasysym}

\usetheme{Warsaw}

\definecolor{Brown}{cmyk}{0,0.81,1,0.60}
\definecolor{OliveGreen}{cmyk}{0.64,0,0.95,0.40}
\definecolor{CadetBlue}{cmyk}{0.62,0.57,0.23,0}
\renewcommand{\ttdefault}{pcr}

\begin{document}

\lstset{
  language=Java,
  basicstyle=\ttfamily\normalsize,
  keywordstyle=\bfseries\color{Brown},
  commentstyle=\color{OliveGreen},
  stringstyle=\color[rgb]{0,0,1},
  tabsize=4
}

\begin{frame}[t,fragile]
%%%%%%%%%%%%%%%%%%%%%%%%%%%%%%%%%%%%%%%%%%%%%%%%%%%%%%%%%%%%%%%%%%%%%%%%%%%%%%%%%%%%%%%%%%%%%%%%%%%%%%%%%%%%%%
\frametitle{What does this program print?}%
\begin{block}{Attempt 1}
\begin{center}%
\begin{lstlisting}
public class Rec {
private static int f(int 
x){
if(x<2){    return 
	1;
}return f(x- 
1)+f(x-2
);}
public static void main(
String[] args) {
System.out.println(f(5));}}
\end{lstlisting}%
\end{center}%
\end{block}
%%%%%%%%%%%%%%%%%%%%%%%%%%%%%%%%%%%%%%%%%%%%%%%%%%%%%%%%%%%%%%%%%%%%%%%%%%%%%%%%%%%%%%%%%%%%%%%%%%%%%%%%%%%%%%
\end{frame}

\begin{frame}[t,fragile]
%%%%%%%%%%%%%%%%%%%%%%%%%%%%%%%%%%%%%%%%%%%%%%%%%%%%%%%%%%%%%%%%%%%%%%%%%%%%%%%%%%%%%%%%%%%%%%%%%%%%%%%%%%%%%%
\frametitle{What does this program print?}%
%
\begin{block}{Attempt 2: A hint\ldots}%
\begin{lstlisting}
public class Rec {
    private static int f(int x) {
		if (x < 2) {
			return 1;
		}
		return f(x - 1) + f(x - 2);
	}

	public static void main(String[] args) {
		System.out.println(f(5));
	}
}
\end{lstlisting}%
\end{block}%
%%%%%%%%%%%%%%%%%%%%%%%%%%%%%%%%%%%%%%%%%%%%%%%%%%%%%%%%%%%%%%%%%%%%%%%%%%%%%%%%%%%%%%%%%%%%%%%%%%%%%%%%%%%%%%
\end{frame}

\begin{frame}[c,fragile]
%%%%%%%%%%%%%%%%%%%%%%%%%%%%%%%%%%%%%%%%%%%%%%%%%%%%%%%%%%%%%%%%%%%%%%%%%%%%%%%%%%%%%%%%%%%%%%%%%%%%%%%%%%%%%%
\frametitle{What does this class do?}%
%
\begin{block}{Attempt 1}%
\begin{lstlisting}
public static final class Oc {
	private final Object[] e 
					= new Object[1000000];
	private int pe = -1;
	private int po = 0;

	public void a(Object x) {
		e[po++] = x;
	}

	public Object b() {
		return e[pe++];
	}
}
\end{lstlisting}%
\end{block}%

%%%%%%%%%%%%%%%%%%%%%%%%%%%%%%%%%%%%%%%%%%%%%%%%%%%%%%%%%%%%%%%%%%%%%%%%%%%%%%%%%%%%%%%%%%%%%%%%%%%%%%%%%%%%%%
\end{frame}

\begin{frame}[c,fragile]
%%%%%%%%%%%%%%%%%%%%%%%%%%%%%%%%%%%%%%%%%%%%%%%%%%%%%%%%%%%%%%%%%%%%%%%%%%%%%%%%%%%%%%%%%%%%%%%%%%%%%%%%%%%%%%
\frametitle{What does this class do?}%
%
\begin{block}{Attempt 2: A hint\ldots}
\begin{lstlisting}
public static final class Queue {
	private final Object[] myValues 
					= new Object[BIG_VALUE];
	private int myHead = -1;
	private int myTail = 0;
	
	public void enqueue(Object x) {
		myValues[myTail++] = x;
	}
	
	public Object dequeue() {
		return myValues[myHead++];
	}
}
\end{lstlisting}%
\end{block}
%%%%%%%%%%%%%%%%%%%%%%%%%%%%%%%%%%%%%%%%%%%%%%%%%%%%%%%%%%%%%%%%%%%%%%%%%%%%%%%%%%%%%%%%%%%%%%%%%%%%%%%%%%%%%%
\end{frame}

\begin{frame}[c,fragile]
%%%%%%%%%%%%%%%%%%%%%%%%%%%%%%%%%%%%%%%%%%%%%%%%%%%%%%%%%%%%%%%%%%%%%%%%%%%%%%%%%%%%%%%%%%%%%%%%%%%%%%%%%%%%%%
\frametitle{A bug fixed}%
%
\begin{block}{class Queue}
\begin{lstlisting}
	private int myHead = -1;
	private int myTail = 0;
	
	public void enqueue(Object x) {
		myValues[myTail] = x;
		myTail++;
	}
	
	public Object dequeue() {
		myHead++;
		return myValues[myHead];
	}
\end{lstlisting}%
\end{block}
%%%%%%%%%%%%%%%%%%%%%%%%%%%%%%%%%%%%%%%%%%%%%%%%%%%%%%%%%%%%%%%%%%%%%%%%%%%%%%%%%%%%%%%%%%%%%%%%%%%%%%%%%%%%%%
\end{frame}

\begin{frame}[t,fragile]
%%%%%%%%%%%%%%%%%%%%%%%%%%%%%%%%%%%%%%%%%%%%%%%%%%%%%%%%%%%%%%%%%%%%%%%%%%%%%%%%%%%%%%%%%%%%%%%%%%%%%%%%%%%%%%
\frametitle{What does this function do?}%
%
\begin{block}{A humble two-line function \smiley}
\begin{lstlisting}
boolean p(int x) {
	int y = x * (030 >> 4 << 030);
	return y == 0;
}
\end{lstlisting}%		
\end{block}

\begin{block}{Hints}<2->
	\begin{itemize}
		\item \texttt{a << s = a * $\mathtt{2^s}$} --- bitwise shift left
		\item \texttt{a >> s = a / $\mathtt{2^s}$} --- bitwise shift right
		\item \texttt{0x<DIGITS>} --- hexadecimal number
		\item \texttt{0<DIGITS>} --- octal number
	\end{itemize}
\end{block}
%%%%%%%%%%%%%%%%%%%%%%%%%%%%%%%%%%%%%%%%%%%%%%%%%%%%%%%%%%%%%%%%%%%%%%%%%%%%%%%%%%%%%%%%%%%%%%%%%%%%%%%%%%%%%%
\end{frame}

\begin{frame}[fragile]
%%%%%%%%%%%%%%%%%%%%%%%%%%%%%%%%%%%%%%%%%%%%%%%%%%%%%%%%%%%%%%%%%%%%%%%%%%%%%%%%%%%%%%%%%%%%%%%%%%%%%%%%%%%%%%
\frametitle{What we have (hopefully) learned so far}%
%
\begin{block}{Example 1: Fibonacci numbers}<2->
	Format your programs properly!
\end{block}
\begin{block}{Example 2: Queue}<3->
	Give understandable names to program elements!\\
	A good program does not demand comments other than JavaDoc for interfaces.
\end{block}
\begin{block}{Example 3: Divisibility by 256}<4->
	Do not outsmart yourself!\\
	\only<5->{Use understandable code constructs.}
\end{block}

%%%%%%%%%%%%%%%%%%%%%%%%%%%%%%%%%%%%%%%%%%%%%%%%%%%%%%%%%%%%%%%%%%%%%%%%%%%%%%%%%%%%%%%%%%%%%%%%%%%%%%%%%%%%%%
\end{frame}
\end{document}
