\documentclass[paper=screen,10pt,unicode]{beamer}
\usepackage[T2A]{fontenc}
\usepackage[utf8]{inputenc}
\usepackage[russian, english]{babel}

\mode<presentation>
{
	\usetheme{Frankfurt}
}

\author{Андрей Бреслав \\ \texttt{abreslav@gmail.com}}
\institute[ИТМО]{СПбГУ ИТМО}

\subject{Разработка технологии создания программного обеспечения,
основанного на формальных грамматиках, поддерживающей
модульность и повторное использование}
\AtBeginSection[]
{
	\begin{frame}<beamer>
		\frametitle{Содержание}
		\tableofcontents[currentsection,currentsubsection]
	\end{frame}
}

\title[Grammarware Engineering]{Разработка технологии создания программного обеспечения,
основанного на формальных грамматиках, поддерживающей
модульность и повторное использование}

\date{\today}

\begin{document}

\begin{frame}
	\titlepage
\end{frame}

\begin{frame}
	\frametitle{Содержание}
	\tableofcontents
\end{frame}

\section{Цели и задачи}
\begin{frame}
	\frametitle{Цели и задачи}

	\begin{block}{Цели проекта}
		Создание технологии разработки ПО, связанного с грамматиками, 
		обеспечивающей внутреннее качество создаваемого ПО:
		\begin{itemize}
			\item модульность
			\item повторное использование
			\item разделение раличных аспектов системы
		\end{itemize}
	\end{block}
	
	\begin{block}{Задачи}
		\begin{itemize}
			\item Формализация процесса разработки в виде последовательности преобразований грамматики
			\item Разработка унифицированнго языка описания грамматик
			\item Разработка языка запросов к грамматикам
			\item Разработка языков описания преобразований
			\item Разработка библиотеки повторно используемых описаний и преобразований
		\end{itemize}
	\end{block}
\end{frame}
\section{Описание проблемы}
\begin{frame}
	\frametitle{ПО, связанное с грамматиками (Grammarware)}
	\begin{block}{Приложения}
		\begin{itemize}
			\item Компиляторы
			\item Средства статического анализа кода
			\item Форматы файлов данных
			\item Средства преобразования кода
			\item Средства автоматического документирования
			\item ...
		\end{itemize}
	\end{block}
	\begin{block}{Средства разработки}
		\begin{itemize}
			\item Генераторы компиляторов (интерпретаторов, парсеров)
			\item Генераторы текстовых редакторов с подсветкой
			\item \large \bf\alert{???}
		\end{itemize}
	\end{block}
\end{frame}

\begin{frame}
	\frametitle{Проблемы, связанные с разработкой Grammarware}

	\begin{block}{С грамматическими определениями очень трудно работать}
		\begin{itemize}
			\item Плохая читаемость
			\item Смешение различных функций
			\item Очень трудно поддерживать
			\item Практически невозможно повторно использовать
			\item Большое количество дублирующейся информации
		\end{itemize}
	\end{block}
	\begin{block}{Результат}
		\begin{itemize}
			\item Многие делают все вручную
			\item Возникает большое количество ошибок
			\item Создание качественного ПО, работающего с грамматиками, затруднено
		\end{itemize}
	\end{block}
\end{frame}

\begin{frame}
	\frametitle{Grammarware Engineering}

	\begin{block}{Основатели подхода}
		Статья P. Klint, R. L\"{a}ammel, C. Verhoef, ``Towards an engineering discipline for GRAMMARWARE'', 2005 год
	\end{block}
	\begin{block}{}
		\alert{Grammarware} -- программы, использующие грамматики или основанные на грамматиках.
	\end{block}
	\begin{block}{Grammarware engineering}
		\begin{itemize}
			\item Шаблоны проектирования
			\item Стандарты
			\item {\bf Специализированные средства разработки }
		\end{itemize}
	\end{block}
\end{frame}

\section{Предлагаемый подход}
\begin{frame}
	\frametitle{Что такое Grammatic}

	\begin{block}{Идея}
		Программное средство для разработки Grammarware, основанное на идеях Model-Driven Development
		и Аспектно-Ориентированного Программирования
	\end{block}
	\begin{block}{Преимущества}
		\begin{itemize}
			\item Разделение аспектов системы
			\item Модульность и повторное исопльзование артефактов разработки
			\item Гибкость при эволюции системы
			\item Выбор конкретной технологии (класс грамматики, платформа и т.д.) на поздних этапах разработки
			\item Автоматизация манипуляций с грамматиками
		\end{itemize}
	\end{block}
\end{frame}

\begin{frame}
	\frametitle{Как этого добиться}

	\begin{block}{Определение грамматики ничем не расширено}
		\begin{itemize}
			\item ``Чистая'' EBNF
			\item Никакого кода семантических действий
			\item Никаких дополнительных ограничений на класс грамматики (LL(k), LALR и т.д.)
		\end{itemize}
	\end{block}
	\begin{block}{Прочие аспекты системы определяются отдельно}
		\begin{itemize}
			\item Проверка грамматики на корректность
			\item Преобразования грамматики
			\item Генерация моделей по грамматике
			\item Генерация кода
			\item ...
		\end{itemize}
	\end{block}
\end{frame}

\begin{frame}
	\frametitle{Реализация}
	\begin{block}{Запросы и преобразования}
		\begin{itemize}
			\item Выбрать из грамматики объекты с заданными свойствами
			\item Преобразовать выбранные объекты
		\end{itemize}
	\end{block}
	\begin{block}{}
		Примеры запросов
		\begin{itemize}
			\item Все нетерминалы, имена которых начинаются на E
			\item Все правила, у которых в правой части по два нетерминала
		\end{itemize}
	\end{block}
	\begin{block}{Преобразования}
		\begin{itemize}
			\item Изменение структуры грамматики
			\item Создание объектов в какой-либо метамодели
			\item Генерация сообщений об ошибках
			\item Генерация текста
		\end{itemize}
	\end{block}
\end{frame}

\begin{frame}
	\frametitle{Можно создавать специализированные инструменты}

	\begin{block}{Примеры специализированных инструментов}
		\begin{itemize}
			\item Генератор парсеров
			\item Генератор тестов для грамматик
			\item Преобразователь грамматики в UML-диаграмму классов
		\end{itemize}
	\end{block}
	\begin{block}{Составляющие специализированного инструмента}
		\begin{itemize}
			\item Ограничения на структуру грамматики
			\item Ограничения на метаданные
			\item Преобразования грамматики
		\end{itemize}
	\end{block}
	\begin{block}{Вывод}
		Grammatic может стать платформой для создания таких инструментов
	\end{block}
\end{frame}

\section{Этапы НИОКР}

\begin{frame}
	\frametitle{Этапы НИОКР}

	\begin{block}{}
		\begin{itemize}
			\item Формализация процесса разработки в виде последовательности преобразований грамматики
			\item Разработка модели, описывающей грамматику и модели метaданных
			\item Разработка специализированных языков
			\item Разработка генератора для конкретной платформы (ANTLR/Java)
			\item Разработка специализированного средства на основе созданных инструментов 
			\item Выявление паттернов проектирования и модулей для создания стандартной библиотеки 
		\end{itemize}
	\end{block}
\end{frame}
\section{Результат НИОКР}
\begin{frame}
	\frametitle{Результаты НИОКР}
	\begin{block}{}
		\begin{itemize}
			\item Программные средства, реализующие предлагаемый подход
			\item Примеры использования этих средств на реальных задачах
			\item Паттерны проектирования, применяемые в рамках предложенного подхода
			\item Стандатртная библиотека модулей для быстрого конструирования новых систем
			\item Статьи, призванные популяризовать этот подход
		\end{itemize}
	\end{block}
\end{frame}

\section{Коммерциализуемость}

\begin{frame}
	\frametitle{Коммерциализуемость и дальнейшее развитие}
	\begin{block}{Целевая аудитория}
		\begin{itemize}
			\item Разработчики предметно-ориентированных языков
			\item Разработчики промышленных систем, связанных с грамматиками (компиляторов, средств анализа и т.д.)
		\end{itemize}
	\end{block}
	\begin{block}{Форма распространения}
		\begin{itemize}
			\item Бесплатное ядро*
			\item Коммерческие библиотеки**
				\begin{itemize}
					\item Готовые модули для RAD
					\item Генераторы под конкретные платформы
					\item Front-ends для популярных языков программирования
				\end{itemize}
			\item Коммерческая интегрированная среда разработки (IDE)**
		\end{itemize}
		{\small * - разработано в рамках НИОКР\\
		** - предстоит разработать после окончания НИОКР (возможно, в рамках программы "СТАРТ")
		}
	\end{block}
\end{frame}

\end{document}

