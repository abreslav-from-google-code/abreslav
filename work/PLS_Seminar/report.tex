\documentclass[a4paper,12pt]{article}
\usepackage{underscore}
\usepackage{ifthen}

\newcommand{\Tcg}{\textsc{Tcg}}
\newcommand{\code}[1]{\texttt{#1}}
\newcounter{premise}
\newcommand{\premise}[2]{
\ifthenelse{\equal{\arabic{premise}}{1}}{\\}{}
    \setcounter{premise}{1}
    #1\vdash#2
}
\newcommand{\ifnotempty}[2]{\ifthenelse{\equal{#1}{}}{}{#2}}
\newcommand{\tcgrule}[5]{
	\setcounter{premise}{0}
$$
    \ifthenelse{\equal{#1}{}}{}{
        \forall \left(#1\right)\;
    }
    \begin{array}{c}
	    \begin{array}{l}
	    #2
            \end{array}\\
	    \hline
            #3
    \end{array}\;\ifnotempty{#4}{(\mathtt{#4}\ifnotempty{#5}{[\mathbf{#5}]})}
$$
}
\begin{document}
A languages specification in \Tcg{} covers 
\begin{itemize}
	\item context-free syntax;
	\item abstract syntax tree (AST) construction;
	\item typing rules;
	\item proof search instructions;
	\item pretty-printing rules.
\end{itemize}

Context-free syntax is described by YACC-style productions \cite{YACC}, each grammar must have a start symbol called \code{file}. Productions may be annotated with AST building instructions (after \code{-->} sign) or a block of code to be executed after the production has been matched (in \code{\{!} \ldots \code{!\}} brackets). The \code{file} symbol is responsible for running type analysis, which is usually done by invoking \code{run} command:

\begin{verbatim}
	file: top_wrap EOF {! run ~save: [ (input.base^".rls",
	                               [ ("define", "save_defined") ]) ] $1 !}
\end{verbatim}

The arguments passed to the \code{run} function deal with saving results and are irrelevant for our discussion.

\subsection{Symply-Typed $\lambda$-Calculus}

Following the logic of Chapter 4 of the original work, we will illustrate the main \Tcg{}'s features by examples starting from simply-typed $\lambda$-calculus. Its syntax is described by the following rules (we omit lexical definitions and documenting annotations):

\begin{verbatim}
top_wrap: tops        --> tops($1)
tops: top             --> $1 :: []
    | top ";;" tops   --> $1 :: $3
top: exp              --> $1

exp: ID               --> id[$1]
   | exp exp          --> apply($1,$2)
   | "\\" ID "." exp  --> lambda(id[$2],$4)
   | "(" exp ")"      --> $2
\end{verbatim}

The first three rules describe a sentence as a list of expressions (\code{exp}) separated by double-semicolons (\code{";;"}). The last rule describes a $\lambda$-expression in a usual way: as a variable, application, abstraction or bracketed expression. As mentioned before, instructions after \code{-->} symbols serve for AST-construction; there are several types of such instructions:
 * item references of the form \code{\$<number>} -- reference items on the left-hand side;
 * node constructors of the form \code{<node_type>(<children>)};
 * list constructors of the form \code{head::tail}, where \code{[]} denotes the empty list;
 * opaque term constructors of the form \code{<class>[<value>]} -- these will be explained later.

For example, here is a textual representation of the expression
$$(\lambda f.\lambda x. f\; x)\; g\; y$$
will be\\
\code{(\textbackslash f.\textbackslash x.f x) g x}\\
and an AST constructed from it is
\begin{verbatim}
tops(
    apply(
        apply(
            lambda(id[f], 
                lambda(id[x], 
                    apply(id[f], id[x]))),
            g),
        x)
    ::[]
)
\end{verbatim}

A \emph{typing judgement} in \Tcg{} is a pair 
$$term : type$$
where $term$ is a AST subtree and $type$ is term which structure is defined by typing rules. The typing rules are used by \Tcg{} to construct proofs of typing judgements, which are basically trees where the proven judgement is a root and all the edges are \emph{properly} labeled with rules (what ``properly'' means will be explained later). 

A generic typing rule consists of a set of universally quantified \emph{variables}, a list of \emph{premises}, a \emph{conclusion} and a list of \emph{parameters}. In \Tcg{} concrete syntax it is written like this:
\begin{verbatim}
rule apply
  forall(f,e,s,t)
    apply(f,e) : t
  if f : fun(s,t)
  and e : s
\end{verbatim}

In this example, \code{f}, \code{e}, \code{s} and \code{t} are the quantified variables, \code{apply(f,e)} is the conclusion, \code{f:fun(s,t)} and \code{e:s} are the premises. In more familiar logical notation it can be written like this:
\tcgrule{f, e, s, t}{
    \Gamma \vdash f : s \rightarrow t
    \hspace{20pt}
    \Gamma \vdash e : s
}{\Gamma \vdash f\; e : t}{apply}{}

So this rule expresses the typing for application. To construct a proof for a judgement $f e : t$ we have to build a tree, where this judgement will be the root node and its children will be obtained by \emph{application} of this rule. So the root node will have two children having the forms of the premises, where $f$, $e$ and $t$ are already known (they appear in the initial judgement) and $s$ must be inferred. Each node in the tree is marked with the rule applied at this node, in our case the root node will be marked with \code{apply} rule. To complete the proof we must construct subtrees proving the children, so that leaves of the tree will be marked with application of rules which have no premises. This example provides a rough intuition about proof structure, we will improve it further (for formal definitions see Chapter 2 of the original thesis).

We have written $\Gamma$ in the example above to provide the most familiar notation, but in \Tcg{} the context is ``global'' for the rule application and is being modified each time a rule is applied (the modified version is relevant to the subtree of the vertex marked with the applied rule). This is motivated by the backward style of reasoning adopted by \Tcg{}: when we apply a rule, we create new vertices in the proof tree and change our knowledge about currently available typing information. These changes are expressed by \emph{context modifiers} which are written in the premises instead of context variables like $\Gamma$. For example, here is the rule for $\lambda$-abstraction:
\begin{verbatim}
rule lambda
  forall(x,e,s,t)
    lambda(x,e) : fun(s,t)
  if e : t
    under -( :.1.= x) + [ x : s ]
\end{verbatim}
Here is the tree-like form of this rule:
\tcgrule{x, e, s, t}{
    \premise{-( :_1 = x), +[ x : s ]}{e : t}
}{\lambda x. e : fun(s,t)}{lambda}{}
The context modifiers here are \code{-(:.1. = x)} and \code{+[ x : s ]}. The first one removes rules assigning a type to $x$ from the context of the conclusion, it consists of two parts: a minus sign, which denotes removal and a \emph{selector}
$$:_1 = x$$
which is basically a pattern meaning that the top function symbol in terms we are going to delete is a colon and its first argument is $x$. Another context modifier in this example adds a new typing fact, $x : s$.

The context modifiers are applied to the context available a the root of the subtree, so they are relative to the context of the conclusion, that is why we do not write anything denoting context under the like.

The rule \code{apply} mentioned above has no context modifiers, thus in tree-like form it is written as follows:
\tcgrule{f, e, s, t}{
    \premise{}{f : s \rightarrow t}
    \premise{}{e : s}
}{f e : t}{apply}{}

The third rule for simply-typed $\lambda$-calculus will be the following:
\tcgrule{es, ts}{
\premise{}{\mathtt{[branch]}\; es : ts \; \mathtt{export* ...}}
}{tops(es)}{tops}{}
Note that the conclusion here is not a typing judgement; there is no problem: we are basically proving a theorem and our language is not limited to a single predicate ``:''. The premise contains a \code{[branch]} modifier which denotes that it must be looked for by a independent sub-search procedure; the ellipsis \code{...} is a shorthand denoting list iteration: in our AST structure \code{tops} contains a list of children, so the \code{es} variable will iterate through it item-by-item.

The final spet is to construct an initial set of rules, which are available at the beinning of the proof search.  We put all our rules there:
\begin{verbatim}
environment apply,lambda,tops
\end{verbatim}

\subsection{Introducing Constants}

Assume we want to have integer or boolean literals in our language. These will by represented by corresponding tokens: \code{INT}, \code{``true''} and \code{``false''}. AST productions for these will be the following:
\begin{verbatim}
    INT     --> int[$1]
  | "false" --> bool["false"]
  | "true"  --> bool["true"]
\end{verbatim}

As we have seen before, here \emph{opaque} terms, e.g. \code{int[\$1]}, are constructed instead of normal AST nodes, these terms are opaque for the type checker: it can not look inside them, and can only consider their classes (\code{int} and \code{bool} in our case), but the values can be extracted for output (e.g., for error reporting).

Typing rules for such constants will be of the following form:
\tcgrule{i}{}{\mathtt{int[}i\mathtt{]}:\mathtt{int}}{int\_const}{}

We can declare ``primitive'' or ``built-in'' operations in the same manner:
\tcgrule{}{}{add : int \rightarrow int \rightarrow int}{add\_function}{}

And likewise are conditional expressions:
\tcgrule{e_1,e_2,e_3,t}{
\premise{}{e_1 : bool}
\premise{}{e_2 : t}
\premise{}{e_3 : t}
}{\mathbf{if}\; e_1\; \mathbf{then}\; e_2\; \mathbf{else}\; e_3\; \mathbf{endif} : t}{if\_expr}{}

\subsection{Bindings}

For the monomorphic version of let expression (\code{let x = e in e'}) it is sufficient to treat it as $\lambda$ abstraction applied to the bound term: $(\lambda x.e')\;e$. Thus we can write down the following rule:
\tcgrule{x,e,e',s,t}{
\premise{}{e' : s}
\premise{-(:_1 = x), +[x : s]}{e : t}
}{\mathbf{let}\;x = e'\;\mathbf{in}\;e : t}{mono\_let}{}

In case of polymorphism, which means that $x$ may be typed with $\forall \widetilde{\alpha}.s$, this approach will not work directly, since we need to quantify all the \emph{inner variables} of the type scheme $s$, which are the variable which appear freely in the proof of $e' : s$ and do not appear in other parts of the proof tree. The operators presented by now do not allow to do this, so \Tcg{} introduces more tools which are expressive enough to cover it (and much more): \emph{rule extraction} and \emph{forward application}. To illustrate these techniques, we will first reformulate the above rule \code{mono\_let} and then extend it to a polymorphic version.

Rule extraction basically just takes a (possibly incomplete) subproof and turns it into a new rule: the root of the subproof becomes a conclusion, and the leafs become premises. If the subproof was complete, then there will be no premises in the rule (all the premises will be empty). This operation corresponds to the idea of memorizing subproofs as lemmata and re-using them in different parts of the proof. Thus we are going to extract a proof of $e' : s$ and then use it in the proof for the second premise (and later for the proof of $\forall x : \widetilde{\alpha}.s$).

By now we can easily prove $e' : s$, but what we need for the second premise is $x : s$. This is where forward application comes into the stage: it allows to combine two rules into one if the premise of the second rule unify with the conclusion of the first one.
$$
\left.\begin{array}{r}
	\tcgrule{}{\ldots}{e' : s}{}{}\\
	+\hspace{14pt}\\
	\tcgrule{e,t}{e : t}{x : t}{}{}
\end{array}
\right\}
\Rightarrow 
\begin{array}{l}
	\tcgrule{}{\ldots}{x : t}{}{}
\end{array}
$$
This operation easily generalizes to more than one premise of the second rule.

Now we can reformulate \code{mono_let} in this manner: in the context modifier for the second premise, we will extract the proof of the first premise and apply the forward reasoning.

\tcgrule{x,e,e',s,t}{
    \premise{}{e' : s}
    \premise{-(:_1 = x), +[\mathtt{let\_binding}[x]](\langle 1 \rangle)}{e : t}
}{\mathbf{let}\;x = e'\;\mathbf{in}\;e : t}{let\_subproof}{}

The auxiliary rule \code{let_binding} looks like this:
\tcgrule{e,t}{\premise{}{e : t}}{\mathbf{y} : t}{let\_binding}{y}

Let us explain the new pieces of notation now. First, we have a \emph{rule reference} in the context modifier of the first premise:
$$\mathtt{let\_binding}[x]$$
this simply adds the rule \code{let_binding} to the context, passing $x$ to it as an argument. As you can see, the rule has a formal parameter $\mathbf{y}$; this parameter must be instantiated with a term upon the rule application; in our case it is instantiated with the term $x$.

The rule reference is followed by a forward application:
$$[\mathtt{let\_binding}[x]](\langle 1 \rangle)$$
this notation means ``extract the proof of the first premise and apply forward reasoning to it an the rule $\mathtt{let\_binding}[x]$''. The generic syntax for this operation will be
$$rule\_expression(rule\_expressions)$$
where each rule expression may be 
\begin{itemize}
	\item an inline rule (e.g., $x : s$), 
	\item a reference (e.g., $\mathtt{let\_binding}[x]$), 
	\item a rule extracted from the subproof of the premise number $i$ (e.g., $\langle 1 \rangle$, the premise must stay to the left of the current premise),
	\item the \textbf{environment} which refers to all the rules present in the context.
\end{itemize}

Now we proceed to the polymorphic version of \textbf{let}. To complete the definition, we change \code{let_subproof} only slightly:

\tcgrule{x,e,e',s,t}{
    \premise{}{e' : s}
    \premise{-(:_1 = x), +[\mathtt{let\_binding}[x]](\langle 1 : [\forall] \rangle)}{e : t}
}{\mathbf{let}\;x = e'\;\mathbf{in}\;e : t}{let\_poly}{}

Here $\langle 1 : [\forall] \rangle$ stands for ``extract the subproof of the first premise and quantify its inner variables universally''. These will be exactly $\widetilde{\alpha}$ --- the variables quantified in the type scheme $\forall \widetilde{\alpha}.s$. Note that the type scheme itself does not appear in the rule: we do not need to enrich our predicate language with universal quantification since it is successfully handled by the meta-theory. Why is the meta-theory designed in such a way that is serves exactly this purpose? The author of the original work gives a detailed explanation in the Section 4.1.3.3, to put it shortly: quantifying the free variables is a natural operation which makes a rule usable, and inner variables are the maximal set of variables which can be quantified without affecting soundness.

\subsection{Recursion}

To define non-trivial functions with \textbf{let}, we need recursion, which is not yet supported by our rule. To support it we can just add a context modifier to the first premise:

\tcgrule{x,e,e',s,t}{
    \premise{+[x : s]}{e' : s}
    \premise{-(:_1 = x), +[\mathtt{let\_binding}[x]](\langle 1 : [\forall] \rangle)}{e : t}
}{\mathbf{let}\;x = e'\;\mathbf{in}\;e : t}{let\_poly}{}

By doing this we have added the assumption $x : s$ to the proof for $e' : s$, and since $x$ is being bound to $e'$ this indeed enables recursion. By now we can handle basic functionality available in \textsc{MiniML} language \cite{MiniML}. In further sections we illustrate more sophisticated features of \Tcg{} by extending the capabilities of the language. Among others, we will cover tuples and mutual recursion.

\subsection{Tuples}
\end{document}

