
%% bare_conf.tex
%% V1.3
%% 2007/01/11
%% by Michael Shell
%% See:
%% http://www.michaelshell.org/
%% for current contact information.
%%
%% This is a skeleton file demonstrating the use of IEEEtran.cls
%% (requires IEEEtran.cls version 1.7 or later) with an IEEE conference paper.
%%
%% Support sites:
%% http://www.michaelshell.org/tex/ieeetran/
%% http://www.ctan.org/tex-archive/macros/latex/contrib/IEEEtran/
%% and
%% http://www.ieee.org/

%%*************************************************************************
%% Legal Notice:
%% This code is offered as-is without any warranty either expressed or
%% implied; without even the implied warranty of MERCHANTABILITY or
%% FITNESS FOR A PARTICULAR PURPOSE! 
%% User assumes all risk.
%% In no event shall IEEE or any contributor to this code be liable for
%% any damages or losses, including, but not limited to, incidental,
%% consequential, or any other damages, resulting from the use or misuse
%% of any information contained here.
%%
%% All comments are the opinions of their respective authors and are not
%% necessarily endorsed by the IEEE.
%%
%% This work is distributed under the LaTeX Project Public License (LPPL)
%% ( http://www.latex-project.org/ ) version 1.3, and may be freely used,
%% distributed and modified. A copy of the LPPL, version 1.3, is included
%% in the base LaTeX documentation of all distributions of LaTeX released
%% 2003/12/01 or later.
%% Retain all contribution notices and credits.
%% ** Modified files should be clearly indicated as such, including  **
%% ** renaming them and changing author support contact information. **
%%
%% File list of work: IEEEtran.cls, IEEEtran_HOWTO.pdf, bare_adv.tex,
%%                    bare_conf.tex, bare_jrnl.tex, bare_jrnl_compsoc.tex
%%*************************************************************************

% *** Authors should verify (and, if needed, correct) their LaTeX system  ***
% *** with the testflow diagnostic prior to trusting their LaTeX platform ***
% *** with production work. IEEE's font choices can trigger bugs that do  ***
% *** not appear when using other class files.                            ***
% The testflow support page is at:
% http://www.michaelshell.org/tex/testflow/



% Note that the a4paper option is mainly intended so that authors in
% countries using A4 can easily print to A4 and see how their papers will
% look in print - the typesetting of the document will not typically be
% affected with changes in paper size (but the bottom and side margins will).
% Use the testflow package mentioned above to verify correct handling of
% both paper sizes by the user's LaTeX system.
%
% Also note that the "draftcls" or "draftclsnofoot", not "draft", option
% should be used if it is desired that the figures are to be displayed in
% draft mode.
%
\documentclass[conference]{IEEEtran}

% Add the compsoc option for Computer Society conferences.
%
% If IEEEtran.cls has not been installed into the LaTeX system files,
% manually specify the path to it like:
% \documentclass[conference]{../sty/IEEEtran}





% Some very useful LaTeX packages include:
% (uncomment the ones you want to load)


% *** MISC UTILITY PACKAGES ***
%
%\usepackage{ifpdf}
% Heiko Oberdiek's ifpdf.sty is very useful if you need conditional
% compilation based on whether the output is pdf or dvi.
% usage:
% \ifpdf
%   % pdf code
% \else
%   % dvi code
% \fi
% The latest version of ifpdf.sty can be obtained from:
% http://www.ctan.org/tex-archive/macros/latex/contrib/oberdiek/
% Also, note that IEEEtran.cls V1.7 and later provides a builtin
% \ifCLASSINFOpdf conditional that works the same way.
% When switching from latex to pdflatex and vice-versa, the compiler may
% have to be run twice to clear warning/error messages.






% *** CITATION PACKAGES ***
%
%\usepackage{cite}
% cite.sty was written by Donald Arseneau
% V1.6 and later of IEEEtran pre-defines the format of the cite.sty package
% \cite{} output to follow that of IEEE. Loading the cite package will
% result in citation numbers being automatically sorted and properly
% "compressed/ranged". e.g., [1], [9], [2], [7], [5], [6] without using
% cite.sty will become [1], [2], [5]--[7], [9] using cite.sty. cite.sty's
% \cite will automatically add leading space, if needed. Use cite.sty's
% noadjust option (cite.sty V3.8 and later) if you want to turn this off.
% cite.sty is already installed on most LaTeX systems. Be sure and use
% version 4.0 (2003-05-27) and later if using hyperref.sty. cite.sty does
% not currently provide for hyperlinked citations.
% The latest version can be obtained at:
% http://www.ctan.org/tex-archive/macros/latex/contrib/cite/
% The documentation is contained in the cite.sty file itself.
% *** GRAPHICS RELATED PACKAGES ***
%
\ifCLASSINFOpdf
  % \usepackage[pdftex]{graphicx}
  % declare the path(s) where your graphic files are
  % \graphicspath{{../pdf/}{../jpeg/}}
  % and their extensions so you won't have to specify these with
  % every instance of \includegraphics
  % \DeclareGraphicsExtensions{.pdf,.jpeg,.png}
\else
  % or other class option (dvipsone, dvipdf, if not using dvips). graphicx
  % will default to the driver specified in the system graphics.cfg if no
  % driver is specified.
  % \usepackage[dvips]{graphicx}
  % declare the path(s) where your graphic files are
  % \graphicspath{{../eps/}}
  % and their extensions so you won't have to specify these with
  % every instance of \includegraphics
  % \DeclareGraphicsExtensions{.eps}
\fi
% graphicx was written by David Carlisle and Sebastian Rahtz. It is
% required if you want graphics, photos, etc. graphicx.sty is already
% installed on most LaTeX systems. The latest version and documentation can
% be obtained at: 
% http://www.ctan.org/tex-archive/macros/latex/required/graphics/
% Another good source of documentation is "Using Imported Graphics in
% LaTeX2e" by Keith Reckdahl which can be found as epslatex.ps or
% epslatex.pdf at: http://www.ctan.org/tex-archive/info/
%
% latex, and pdflatex in dvi mode, support graphics in encapsulated
% postscript (.eps) format. pdflatex in pdf mode supports graphics
% in .pdf, .jpeg, .png and .mps (metapost) formats. Users should ensure
% that all non-photo figures use a vector format (.eps, .pdf, .mps) and
% not a bitmapped formats (.jpeg, .png). IEEE frowns on bitmapped formats
% which can result in "jaggedy"/blurry rendering of lines and letters as
% well as large increases in file sizes.
%
% You can find documentation about the pdfTeX application at:
% http://www.tug.org/applications/pdftex





% *** MATH PACKAGES ***
%
\usepackage[cmex10]{amsmath}
% A popular package from the American Mathematical Society that provides
% many useful and powerful commands for dealing with mathematics. If using
% it, be sure to load this package with the cmex10 option to ensure that
% only type 1 fonts will utilized at all point sizes. Without this option,
% it is possible that some math symbols, particularly those within
% footnotes, will be rendered in bitmap form which will result in a
% document that can not be IEEE Xplore compliant!
%
% Also, note that the amsmath package sets \interdisplaylinepenalty to 10000
% thus preventing page breaks from occurring within multiline equations. Use:
%\interdisplaylinepenalty=2500
% after loading amsmath to restore such page breaks as IEEEtran.cls normally
% does. amsmath.sty is already installed on most LaTeX systems. The latest
% version and documentation can be obtained at:
% http://www.ctan.org/tex-archive/macros/latex/required/amslatex/math/





% *** SPECIALIZED LIST PACKAGES ***
%
%\usepackage{algorithmic}
% algorithmic.sty was written by Peter Williams and Rogerio Brito.
% This package provides an algorithmic environment fo describing algorithms.
% You can use the algorithmic environment in-text or within a figure
% environment to provide for a floating algorithm. Do NOT use the algorithm
% floating environment provided by algorithm.sty (by the same authors) or
% algorithm2e.sty (by Christophe Fiorio) as IEEE does not use dedicated
% algorithm float types and packages that provide these will not provide
% correct IEEE style captions. The latest version and documentation of
% algorithmic.sty can be obtained at:
% http://www.ctan.org/tex-archive/macros/latex/contrib/algorithms/
% There is also a support site at:
% http://algorithms.berlios.de/index.html
% Also of interest may be the (relatively newer and more customizable)
% algorithmicx.sty package by Szasz Janos:
% http://www.ctan.org/tex-archive/macros/latex/contrib/algorithmicx/




% *** ALIGNMENT PACKAGES ***
%
%\usepackage{array}
% Frank Mittelbach's and David Carlisle's array.sty patches and improves
% the standard LaTeX2e array and tabular environments to provide better
% appearance and additional user controls. As the default LaTeX2e table
% generation code is lacking to the point of almost being broken with
% respect to the quality of the end results, all users are strongly
% advised to use an enhanced (at the very least that provided by array.sty)
% set of table tools. array.sty is already installed on most systems. The
% latest version and documentation can be obtained at:
% http://www.ctan.org/tex-archive/macros/latex/required/tools/


%\usepackage{mdwmath}
%\usepackage{mdwtab}
% Also highly recommended is Mark Wooding's extremely powerful MDW tools,
% especially mdwmath.sty and mdwtab.sty which are used to format equations
% and tables, respectively. The MDWtools set is already installed on most
% LaTeX systems. The lastest version and documentation is available at:
% http://www.ctan.org/tex-archive/macros/latex/contrib/mdwtools/


% IEEEtran contains the IEEEeqnarray family of commands that can be used to
% generate multiline equations as well as matrices, tables, etc., of high
% quality.


%\usepackage{eqparbox}
% Also of notable interest is Scott Pakin's eqparbox package for creating
% (automatically sized) equal width boxes - aka "natural width parboxes".
% Available at:
% http://www.ctan.org/tex-archive/macros/latex/contrib/eqparbox/





% *** SUBFIGURE PACKAGES ***
%\usepackage[tight,footnotesize]{subfigure}
% subfigure.sty was written by Steven Douglas Cochran. This package makes it
% easy to put subfigures in your figures. e.g., "Figure 1a and 1b". For IEEE
% work, it is a good idea to load it with the tight package option to reduce
% the amount of white space around the subfigures. subfigure.sty is already
% installed on most LaTeX systems. The latest version and documentation can
% be obtained at:
% http://www.ctan.org/tex-archive/obsolete/macros/latex/contrib/subfigure/
% subfigure.sty has been superceeded by subfig.sty.



%\usepackage[caption=false]{caption}
%\usepackage[font=footnotesize]{subfig}
% subfig.sty, also written by Steven Douglas Cochran, is the modern
% replacement for subfigure.sty. However, subfig.sty requires and
% automatically loads Axel Sommerfeldt's caption.sty which will override
% IEEEtran.cls handling of captions and this will result in nonIEEE style
% figure/table captions. To prevent this problem, be sure and preload
% caption.sty with its "caption=false" package option. This is will preserve
% IEEEtran.cls handing of captions. Version 1.3 (2005/06/28) and later 
% (recommended due to many improvements over 1.2) of subfig.sty supports
% the caption=false option directly:
%\usepackage[caption=false,font=footnotesize]{subfig}
%
% The latest version and documentation can be obtained at:
% http://www.ctan.org/tex-archive/macros/latex/contrib/subfig/
% The latest version and documentation of caption.sty can be obtained at:
% http://www.ctan.org/tex-archive/macros/latex/contrib/caption/




% *** FLOAT PACKAGES ***
%
%\usepackage{fixltx2e}
% fixltx2e, the successor to the earlier fix2col.sty, was written by
% Frank Mittelbach and David Carlisle. This package corrects a few problems
% in the LaTeX2e kernel, the most notable of which is that in current
% LaTeX2e releases, the ordering of single and double column floats is not
% guaranteed to be preserved. Thus, an unpatched LaTeX2e can allow a
% single column figure to be placed prior to an earlier double column
% figure. The latest version and documentation can be found at:
% http://www.ctan.org/tex-archive/macros/latex/base/



%\usepackage{stfloats}
% stfloats.sty was written by Sigitas Tolusis. This package gives LaTeX2e
% the ability to do double column floats at the bottom of the page as well
% as the top. (e.g., "\begin{figure*}[!b]" is not normally possible in
% LaTeX2e). It also provides a command:
%\fnbelowfloat
% to enable the placement of footnotes below bottom floats (the standard
% LaTeX2e kernel puts them above bottom floats). This is an invasive package
% which rewrites many portions of the LaTeX2e float routines. It may not work
% with other packages that modify the LaTeX2e float routines. The latest
% version and documentation can be obtained at:
% http://www.ctan.org/tex-archive/macros/latex/contrib/sttools/
% Documentation is contained in the stfloats.sty comments as well as in the
% presfull.pdf file. Do not use the stfloats baselinefloat ability as IEEE
% does not allow \baselineskip to stretch. Authors submitting work to the
% IEEE should note that IEEE rarely uses double column equations and
% that authors should try to avoid such use. Do not be tempted to use the
% cuted.sty or midfloat.sty packages (also by Sigitas Tolusis) as IEEE does
% not format its papers in such ways.





% *** PDF, URL AND HYPERLINK PACKAGES ***
%
%\usepackage{url}
% url.sty was written by Donald Arseneau. It provides better support for
% handling and breaking URLs. url.sty is already installed on most LaTeX
% systems. The latest version can be obtained at:
% http://www.ctan.org/tex-archive/macros/latex/contrib/misc/
% Read the url.sty source comments for usage information. Basically,
% \url{my_url_here}.





% *** Do not adjust lengths that control margins, column widths, etc. ***
% *** Do not use packages that alter fonts (such as pslatex).         ***
% There should be no need to do such things with IEEEtran.cls V1.6 and later.
% (Unless specifically asked to do so by the journal or conference you plan
% to submit to, of course. )

\usepackage{enumerate}

% correct bad hyphenation here
\hyphenation{op-tical net-works semi-conduc-tor}


\newcommand{\FORMULA}{\textsc{Formula}}

\begin{document}
%
% paper title
% can use linebreaks \\ within to get better formatting as desired
\title{Generalized Design Space Exploration for Optimizing Embedded Software Architectures}


% author names and affiliations
% use a multiple column layout for up to three different
% affiliations
\author{\IEEEauthorblockN{Andrey Breslav}
\IEEEauthorblockA{Department of Mathematics\\
ITMO University\\
St. Petersburg, Russia\\
abreslav@gmail.com}
\and
\IEEEauthorblockN{Ethan K. Jackson}
\IEEEauthorblockA{Research in Software Engineering\\
Microsoft Research\\
Redmond, WA, USA\\
ejackson@microsoft.com}}

% conference papers do not typically use \thanks and this command
% is locked out in conference mode. If really needed, such as for
% the acknowledgment of grants, issue a \IEEEoverridecommandlockouts
% after \documentclass

% for over three affiliations, or if they all won't fit within the width
% of the page, use this alternative format:
% 
%\author{\IEEEauthorblockN{Michael Shell\IEEEauthorrefmark{1},
%Homer Simpson\IEEEauthorrefmark{2},
%James Kirk\IEEEauthorrefmark{3}, 
%Montgomery Scott\IEEEauthorrefmark{3} and
%Eldon Tyrell\IEEEauthorrefmark{4}}
%\IEEEauthorblockA{\IEEEauthorrefmark{1}School of Electrical and Computer Engineering\\
%Georgia Institute of Technology,
%Atlanta, Georgia 30332--0250\\ Email: see http://www.michaelshell.org/contact.html}
%\IEEEauthorblockA{\IEEEauthorrefmark{2}Twentieth Century Fox, Springfield, USA\\
%Email: homer@thesimpsons.com}
%\IEEEauthorblockA{\IEEEauthorrefmark{3}Starfleet Academy, San Francisco, California 96678-2391\\
%Telephone: (800) 555--1212, Fax: (888) 555--1212}
%\IEEEauthorblockA{\IEEEauthorrefmark{4}Tyrell Inc., 123 Replicant Street, Los Angeles, California 90210--4321}}




% use for special paper notices
%\IEEEspecialpapernotice{(Invited Paper)}




% make the title area
\maketitle


\begin{abstract}
%\boldmath
The abstract goes here.
\end{abstract}
% IEEEtran.cls defaults to using nonbold math in the Abstract.
% This preserves the distinction between vectors and scalars. However,
% if the conference you are submitting to favors bold math in the abstract,
% then you can use LaTeX's standard command \boldmath at the very start
% of the abstract to achieve this. Many IEEE journals/conferences frown on
% math in the abstract anyway.

% no keywords




% For peer review papers, you can put extra information on the cover
% page as needed:
% \ifCLASSOPTIONpeerreview
% \begin{center} \bfseries EDICS Category: 3-BBND \end{center}
% \fi
%
% For peerreview papers, this IEEEtran command inserts a page break and
% creates the second title. It will be ignored for other modes.
\IEEEpeerreviewmaketitle

\section{Introduction}
\textit{Design space exploration} (DSE) is a technique for automatically enumerating variants of a system; efficient DSE enables automated rapid prototyping and optimization. Consequently, DSE has been heavily investigated for hardware [,,] and software architectures [,,], yielding encouraging results. 

However, there is a fundamental limitation to many DSE approaches: They explore small subsets of possible architectures by varying a few system parameters. For example, a design space may be constructed by varying some details of a point design, such as the cache sizes of processors [] or the software implementations of FFTs []. The resulting design space contains a family of related point designs differing only along these predefined axes. This restricts optimization to a narrow range of variants derived from a particular point design. 

In this paper we present a novel DSE approach that explores complex variations while supporting search-based optimization. Our approach works in the following way:
\begin{enumerate}
\item \textit{Define.} The design space is defined through constraints written in first-order logic extended with a fixpoint operator. The solutions to these constraints equals the design space. The engineer also provides a ranking function over points in the design space. 
\item \textit{Sample.} Points in the design space are randomly sampled by composing a random sampling process with a \textit{satisfiability modulo theories} (SMT) solver. This provides a sampling procedure that, in practice, efficiently finds diverse points.
\item \textit{Search.} Many points are sampled and ranked. Machine learning estimates those design choices that increase observed ranks, and then good design choices are converted back to first-order constraints. In the next round, variants exhibiting these choices are enumerated and ranked. The process continues to refine the design space as long as observed ranks correlate with estimates, and backtracks on design decisions otherwise.
\end{enumerate}

We illustrate our approach with a case study where DSE explores different data paths of an embedded software architecture and different hardware platforms. The optimization problem takes into account performance estimates of CPU and bus utilization derived from discrete event simulation. The well-known technique of \textit{Naive Bayesian Learning} is employed to estimate the key design choices. 

This paper is organized as follows: 

\section{Related Work}
(Ethan)

\section{Case Study}

The present work was motivated by a case study based on the Skinput project \cite{?} that ``appropriates the human body as an input surface. Skinput resolves the location of finger taps on the arm and hand by analyzing mechanical vibrations that propagate through the body. This approach provides an always available, naturally portable, and on-body finger input system''.

The high-level structure of Skinput is the following: An array of sensors collect vibration amplitudes from the skin and pass this data forward to be classified as a certain gesture. First, a number of features such as average, standard deviation, FFT and center of masses of power spectra, are computed for every sensor. Then, these features are processed by an \mbox{SVM-based} classifier that determines the type of the gesture being observed.

The abstraction we used to model such systems inherits the overall structure from Skinput: We view software as manipulating \emph{values} as inputs and outputs (these model Skinput's features), where some values can be \emph{primary} meaning that they are not computed inside the system (this implicitly models sensors). The values that constitute the output to the user are marked as \emph{target}. 

The software itself is modeled by \emph{programs} that relate input values to output ones. Programs rely on particular \emph{capabilities} of hardware, such as availability of certain instructions.

For example, consider a program $Avg_{DIV}(\overrightarrow{x}, n)$ ($Avg$ stands for ``average'') that computes a value $\sum\limits_{i=0}^n x_i / n$ given two other values: vector $\overrightarrow{x}$ and its length $n$. Assume $Avg_{DIV}$ relies on a hardware implementation of devision; then it requires the following capabilities: $ADD$ and $JMP$ to compute a sum in a loop and $DIV$ to divide it by $n$. On the other hand, a program $Avg_{noDIV}$ that computes the same value but is designed for processors with no built-in division, would require capabilities needed to implement division using simpler instructions.

Hardware platform is modeled as a number of \emph{processors} communicating through \emph{media}. Every hardware element is characterized by its \emph{type}. Processor types specify capabilities the processors provide and medium types they can be connected on.

An \emph{individual design} includes 
\begin{itemize}
\item a hardware topology: processors connected by media,
\item a software configuration: programs that take values as inputs and send resulting values over to other programs,
\item a mapping that assigns programs to be executed on particular processors.
\end{itemize}

\section{Design Spaces as Constraints}

\newcommand{\DataSig}{\Sigma}
\newcommand{\FreeCons}{F}
\newcommand{\Consts}{C}
\newcommand{\Vars}{V}

To formally specify design spaces we used the \FORMULA{} language \cite{??}. In \FORMULA{}, the user defines a signature $\DataSig$ that includes \emph{free constructors}, constants of basic types such as \texttt{Integer} or \texttt{String}, and variables. A design is then represented as a set of terms over $\DataSig$. 

For our case study, we defined a signature corresponding to the abstraction described in the previous section. For example, processors and their capabilities are captured by the following free constructors:

{\scriptsize\begin{verbatim}
    ProcessorType : (name : String).    
    Capability : (name : String).
    CapabilitySupported : (type : ProcessorType, capability : Capability).    
    ConnectionSupported : (ptype : ProcessorType, mtype : MediumType).
    
    MediumType : (name : String).

    Processor : (name : String, type : ProcessorType).
    Medium : (name : String, type : MediumType).    
    Connection : (processor : Processor, medium : Medium).
\end{verbatim}}

The following terms, as a part of a design representation, describe a simple processor type and two processors:    

{\scriptsize\begin{verbatim}
    CPU is ProcessorType("CPU"), // Bound variable
      CapabilitySupported(CPU, Capability("ADD")),
      CapabilitySupported(CPU, Capability("JMP")),
    Processor("cpu1", CPU),
    Processor("cpu2", CPU)
\end{verbatim}}

A \emph{legal} design must fulfill a number of constraints. For example, 
\emph{``processors must be compatible with the media they are connected to''} --- is a typical constraint.
In \FORMULA{}, constraints are expressed as a logic program that takes a set of terms as an input and evaluates to true iff the terms satisfy all the constraints. The following fragment of \FORMULA{} code shows how the constraint mentioned above can be encoded in a logic program:

{\scriptsize\begin{verbatim}
    unsupportedConnection(conn) :- 
        conn is Connection(Processor(_, ptype), Medium(_, mtype)), 
        fail ConnectionSupported(ptype, mtype).
    conforms :? fail unsupportedConnection(_).
\end{verbatim}}

In this example, a term with \texttt{unsupportedConnection} as a root constructor will be created for every connection between a processor and a medium of an unsupported type. The last line expresses the fact that in a legal design no such terms may appear. This is done by defining a special \emph{query} named \texttt{conforms}. A query roughly corresponds to a boolean variable computed by the logic program. The value of the \texttt{conforms} query is considered to be the result of the program written in \FORMULA{}. In our example \texttt{conforms} will be true iff no  \texttt{unsupportedConnection} terms are around.

Other constraints can be expressed in the same manner. To give some examples from software and mapping domains, consider the following constraints:
\begin{itemize}
  \item every target value that is not primary must be computed by some program;
  \item the capabilities required by a program must match the ones available on the processor it is mapped to.
\end{itemize}

For sampling, \FORMULA{} symbolically executes the logic program $P$ representing the constraints and converts it into an SMT-formula $F$, such that e
very satisfying assignment to variables in $F$ corresponds to a set of input terms that makes the \texttt{conforms} query in $P$ true. In order to obtain these assignments, a state-of-the-art SMT solver is applied to $F$.

\section{User-Defined Problem Descriptions}

As we mentioned before, the free constructors in $\DataSig$ describe the kinds of elements that constitute individual designs, and relations between them. The constraints restrict these relations according to what the constructors are intended to mean. These predefined ingredients are common for a wide range of systems as they just capture the overall structure of the system and do not restrict its functionality. 

To describe a particular design problem, such as the one for Skinput, one has to say what designs are \emph{relevant} for this problem. This is done by specifying ``building blocks'' such designs \emph{may} use: particular types of hardware, values and programs to process them. For example, all the user specifies about hardware may be the following (processor types must include with descriptions of their capabilities and supported media):
\begin{itemize}
  \item Processor types: MSP430, PIC32;
  \item Medium types: SPI, I$^2$C;
\end{itemize}
This specification allows all possible topologies connecting processors of types MSP430 and/or PIC32 by media of type SPI and/or I$^2$C in a legal way.

For the software, a set of values and programs manipulating these values must be defined, where every program declares the required capabilities. This may encode many ways of computing the same values. For example, suppose the only primary value is a vector $X$, one other value is the average $E[X]$, and the target value is variance $Var(X)$. In this situation, the user can specify two possible programs computing the target value: first one can directly compute variance as follows:
$$PVar_1(X) = \frac{\sum \left( \sum X_i \middle/ |X| - X_i \right)^2}{|X|},$$
whereas the second one can make use of a pre-computed average value and take it as an input:
$$PVar_2(X, E[X]) = \frac{\sum \left( E[X] - X_i \right)^2}{|X|}.$$
This specification allows two possible \emph{data paths} from $X$ to $Var(X)$: directly or through $E[X]$, and each legal design can contain one of them (our constraints prohibit choosing two paths in the same design). Note that the distinction between different implementations (e.g., requiring built-in division or not) can be applied here as well, producing more possible data paths.

The user provides the ``building blocks'' as a set of terms that is called a \emph{partial model}. Technically, this extends the set of predefined constraints: roughly speaking, it forces the designs to include the terms from the partial model. 



\section{Random Sampling}
(Ethan)

\section{Discussion and Conclusion}


\bibliographystyle{IEEEtran}
\bibliography{DAC2011}
\end{document}


