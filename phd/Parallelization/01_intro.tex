\section{Introduction}

Since Leonard Adleman has published his work \cite{Adleman:1994} in 1994 many additions, improvements and alternative approaches were developed \cite{ Lipton:1994, Amos:2002, Rozenberg:knapsack, Dantsin:2003, Paun:2002}. Several works including \cite{Amos:2002, Amos:2003:Book, PaunRozenbergSalomaa:1998} give good overview of the whole area.

Particularly Adleman's work is a root of the family of computational models working on the same principle: selecting result of the computation from a set of all possible variants (see for example works \cite{Amos:1996, Adleman:1996,  Liu:1996, Rozenberg:2003}). Paper \cite{Katsanyi:2003} shows that these models are computationally equivalent. These models are suitable for solving NP-complete problems.

This approach may be considered to be somewhat extensive since we do not construct the result but just filter out incorrect solutions. Hence algorithms for NP-complete problems in such models require exponential amount of DNA and do not scale to large problem sizes \cite{Hartmanis:1995}. But there is no exact algorithm for NP-complete problem which would have not been exponential. Even probabilistic algorithms require exponential space or time \cite{Liu:2005}. Thus we should not neglect even such ``extensive'' approaches.

Here we deal with the \emph{Parallel Filtering Model} (\emph{PFM}) which is proposed in paper \cite{Amos:1996}. It defines four basic operations:
\begin{itemize}
	\item $Remove(T, \{s_1, \dots, s_n\})$ removes form multiset $T$ any string which contains at least one occurrence of some string $s_i$;
	\item $Union(\{T_1, \dots, T_n\}, T)$ creates a multiset T which is a multiset union of $U_i$;
	\item $Copy(T, \{T_1, \dots, T_n\})$ produces a number of copies, $T_i$, of $T$;
	\item $Select(T)$ selects an element of $T$ uniformly at random, if $T$ is empty then \emph{empty} is returned.
\end{itemize}
Original paper \cite{Amos:1996} assumes each of first three operations to be working \emph{in parallel}, so all the basic operations have time complexity $O(1)$. But in the papers \cite{Amos:1997, Amos:2003:Book} the same authors correct this assumption and mention $Remove$, $Union$ and $Copy$ as having linear time complexity ($O(n)$). 
Paper \cite{Katsanyi:2003} proposes a \emph{normal form} of molecular program in \emph{PFM}, where operations (we will refer to them as \emph{elementary steps}) work with minimum amount of data.
\begin{itemize}
	\item $Union(T_1, T_2, T)$ produces a union $T$ of two multisets $T_1$ and $T_2$. Another form of this operation is $Union(T', T)$ where $T'$ is added to $T$ -- this can be implemented as $Union(T, T', T)$.
	\item $Copy(T, T_1, T_2)$ creates two copies of $T$ ($T$ itself is destroyed then because of the nature of laboratory implementation \cite{Amos:1996}). (This paper proposes simply splitting a tube into two halves, but this way is not applicable in real practice, because we need to duplicate each strand. Paper \cite{Adleman:1994} describe a way to reach real duplication via \emph{PCR}.) Another form of this operation is $Copy(T, T')$ where $T'$ receives a copy of $T$ and $T$ remains existing -- this can be implemented as $Copy(T, T, T')$.
	\item $Remove(T, s)$ (where $s$ is a single string) removes all the strings having a substring $s$ form $T$.
\end{itemize}

Paper \cite{Amos:1997} mention implementations of basic operations using elementary steps implicitly, paper \cite{Katsanyi:2003} defines them explicitly. These implementations are given below ($Copy$ is slightly changed to handle the fact that source tube is destroyed by each elementary $Copy$, which is ignored by \cite{Katsanyi:2003}).

\procedure{$Copy(T, \{T_1, ..., T_n\})$}{
\FOR{$i \gets 1$ \TO $n$}
  \STATE $Copy(T, T_{i+1})$ 
\ENDFOR
}

\procedure{$Union(\{T_1, ..., T_n\}, T)$}{
\STATE $Union(T_1, T_2, T)$
\FOR{$i \gets 3$ \TO $n$}
	\STATE $Union(T_i, T)$
\ENDFOR
}

\procedure{$Remove(T, \{s_1, ..., s_n\})$}{
\FOR{$i \gets 1$ \TO $n$}
	\STATE $Remove(T, s_i)$
\ENDFOR
}

Time complexity of each of these operations is $O(n)$ \cite{Amos:1997, Katsanyi:2003}.

In \cite{Katsanyi:2003} author uses these linear implementations as a foundation of complexity analysis of some algorithms for NP-complete problems which results in polynomial time estimations like $O(n^3)$ for Hamiltonian circuit problem.

In this paper we propose parallel implementations of basic \emph{PFM} operations which lead to $O(log(n))$ time complexities for them. Using such implementations for some algorithms described in \cite{Katsanyi:2003} reduces their execution times from $O(n^3)$ to $O(log(n))$ parallel time.

In section \ref{mpfm} we describe proposed parallel implementation of basic operations. In section \ref{algo} we analyze changes in complexity of some algorithms proposed in \cite{Katsanyi:2003}. The last section \ref{conclusion} gives concluding notes.