\documentclass[a4paper, 12pt]{article}
\usepackage[left=2cm,right=2cm,top=3cm,bottom=3cm]{geometry}
\pagenumbering{none}
\begin{document}
\section*{Andrey Breslav}

\subsection*{Contact Information}
\begin{tabular}{rll}
	Address &\hspace{10pt}& Udarnikov pr., 27-2-237, 195279 Saint-Petersburg, Russia\\
	Phone (mobile) && +7 921 928 67 68 \\
	E-Mail && abreslav@gmail.com \\
	Home page && http://abreslav.googlepages.com\\
\end{tabular}

\subsection*{Research Interests}
\begin{itemize}
	\item Domain-Specific Languages (Textual Syntax) 
	\item Type Systems of Programming Languages
	\item Domain Modeling and Generative Programming
\end{itemize}

\subsection*{Education}

\begin{tabular*}{1.0\textwidth}[t]{p{70pt} l p{370pt}}
	\raggedleft{since 2007}&\hspace{10pt}&PhD student at Mathematics Department, Saint-Petersburg State University of Information Technology, Mechanics and Optics (SPbSU ITMO)\\
	&&Advisor: Prof. Igor Yu. Popov \\
	&&Topic: \textit{Automatically Derived Languages and Applications to DSLs with Aspects}\\
	&&Our goal is to explore the possibilities of, having a description of a DSL,
	  automatic derivation of languages of typed macros (templates) and structural patterns
	  for this DSL. This helps to automate adding support for generic modules and aspects to DSLs. \\
	&&\\
	\raggedleft{2009 -- 2010}&& Visiting PhD student at Institute of Computer Science, University of Tartu, Estonia\\
	&&Advisor: Prof. Marlon Dumas \\
	&&\\
	\raggedleft{2007} && Master's Degree with Honors in Applied Mathematics and Informatics, \\
	&&Computer Technology Department, SPbSU ITMO, GPA 4.90/5.0\\
	&&Thesis: \textit{DSL Development Based on Target Meta-models}\\
	&&The main contribution is a language that allows one to declaratively describe an abstract syntax tree (AST) of a 
	  DSL as a view (in terms of Model-View architecture) of its target meta-model. These descriptions are
	  used as inputs for a generator of translators from mechanically created ASTs trees to target models 
	  (including name resolution). \\
	&&\\
	\raggedleft{2005} && Bachelor's Degree with Honors in Applied Mathematics and Informatics, \\
	&&Computer Technology Department, SPbSU ITMO, GPA 4.96/5.0\\
	&&Thesis: Automated Generation of User Interfaces from Domain Models\\
\end{tabular*}

\subsection*{Teaching}
\begin{tabular}{ r l p{370pt} }
	2006 -- 2009 && {\it Lecturer} at SPbSU ITMO and AMSE\\%
	Courses: && Software Design\\
		 && Introduction to Programming\\
		 && Algorithms and Data Structures\\
		 && Java Programming Language (Basic and Advanced)\\
		 && Software Project (supervision)\\
	summer 2008 && {\it Summer internship supervisor} at OpenWay and SwiftTeams / JetBrains (as a part of AMSE project)\\%
	2003 -- 2009 && {\it Teacher of Informatics} at Lyceum 239, Saint-Petersburg, Russia\\%
\end{tabular}

\subsection*{Thesis supervision}
\begin{tabular}{ r l p{370pt} }
    2009 && Egorov Ivan, St. Petersburg State University\\
         && Specialist (eq. of Master) diploma thesis:
          \it Static Thread Types Analysis for Access Management in Multi-Threaded Java Programs\\
	 &&\\
         && Isakova Svetlana, St. Petersburg Polytechnical University\\
         && Bachelor thesis:
          \it Implementation of Declarative QVT Transformations using Maude\\
	 &&\\
         && Ivanova Alena, SPbSU ITMO\\
         && Bachelor thesis:
          \it Extending JVM Instruction Set with Unchecked Method Calls for Speeding Up Execution on Mobile Platforms\\
	 &&\\
2008     && Rybakov Gleb, SPbSU ITMO\\
         && Bachelor thesis:
          \it Graph-based modelling, Generation and LTL$-$Verification of Data-Oriented User Interfaces\\
\end{tabular}

\subsection*{Software development}
\begin{tabular}{ r l p{395pt} }
	2005 -- 2006 && Borland Labs, Inc., R\verb & D Engineer (Java developer) in Together for Eclipse project, MDA team: work on QVT language implementation, Eclipse plug-in development and integration, EMF modeling, using QVT and XSL/OCL in generative programming samples. \\
2003 && UI development (Delphi) for a suite for automatically planning routs for air photography.\\
2004 && Systems administration in a small hotel (``The Brothers Karamazov hotel'').\\
2004 -- 2005 && Freelance web development (PHP, MySQL).\\
\end{tabular}

\subsection*{Skills}

\begin{tabular}{ r l p{350pt} }
Technical&&Java, Scala, C++, Pascal/Delphi, PHP, JavaScript, SQL, XML, XSLT/XPath, HTML/CSS, \LaTeX, Linux, Windows\\
Languages&&Russian -- native, English -- fluent\\
\end{tabular}

\subsection*{Grants and Scholarships}
\begin{itemize}
	\item DoRa 5 scholarship for visiting PhD studies at University of Tartu from European Regional Development Funds through Archimedes Foundation, 2009
	\item An ``UMNIK'' grant for innovative project development from the {\it FAISE Foundation}, 2009
	\item A grant from {\it The Annual Grant Competition for Young Researchers}, Saint-Petersburg, 2008
	\item {Teaching support grants from 
		\begin{itemize}
			\item Digital Design (2003--2004)  
			\item Borland Labs Inc (2004--2005) 
		\end{itemize}}
\end{itemize}

\subsection*{Publications}
\begin{itemize}
	\item {\it Creating Textual Language Dialects Using Aspect-like Techniques}, In Pre-Proceedings of GTTSE-2009, Braga, 2009 (extended abstract)
	\item {\it Techniques for Formal Grammar Reuse and Their Applications for Creating Dialects}, In Proceedings of KMU-2009, Saint-Petersburg, Russia, April, 2009 (In Russian, abstract in English)
	\item {\it Applying MDD and AOP Principles to Grammarware development}, In Proceedings of KMU-2008, Saint-Petersburg, Russia, April, 2008 (In Russian, abstract in English)
	\item with A. Lukianova, M. Korotkov, {\it Building Class Hierarchies from Informal English Descriptions}, presented at JASS-2004, published in Nauchno-Technicheskiy Vestnik SPbSU ITMO, Vol 39., 2007, pp. 294-303 (In Russian, abstract in English)
	\item with A. Efremov, M. Korotkov, {\it Improving Distance Learning of the Humanities Using {\it RemEd} Online system}, In Proceedings of The 6th IST/IMS-2003 Conference, Saint-Petersburg, Russia, 2003 (In Russian, abstract in English).
\end{itemize}
\end{document}
